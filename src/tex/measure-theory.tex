% This file is included in body.tex

\chapter{Measure Theory}
\label{ch:measure-theory}
%
\section{Integration and Measure}
%
Consider the indicator function
%
\begin{equation*}
  f(x) = \left\{
    \begin{array}{lr}
      0 & : x \in \mathbb{Q}\\
      1 & : x \not\in \mathbb{Q}
    \end{array}
  \right.
\end{equation*}
%
where $\mathbb{Q}$ represents the set of rational numbers (i.e., those
numbers that can be written as a fraction $\sfrac{p}{q}$ with $p$ and
$q$ both integers and $q \neq 0$, or equivalently, those numbers that
can be written as either terminating or repeating decimals.).  Our
goal is to evaluate integrals of the form
%
\begin{equation}
  \label{eq:rational-indicator}
  I = \int_0^1 dx\, f(x)  
\end{equation}
%
The standard approach via Riemann integration is to partition the
interval $[0,1]$ into a finite number of subintervals
$[0 = a_0, a_1], [a_1, a_2], \cdots, [a_{n-1}, a_n = 1]$.
Letting $F_i$ and $f_i$ denote the largest and smallest values of the
function $f(x)$ respectively on the $i$th subinterval
$[a_{i-1}, a_i]$, we then form the upper and lower sums
%
\begin{align*}
  U_n &= \sum_{i=1}^n (a_i - a_{i-1})F_i\\
  L_n &= \sum_{i=1}^n (a_i - a_{i-1})f_i
\end{align*}
%
As this partition is refined (and $n$ increases), $U_n$ decreases to a
limiting value $U$ and $L_n$ increases to a limiting value $L$.
If $U = L$, then the integral is well defined (in the Riemann sense) and has
the value $I = U = L$.  If $U \neq L$, then the Riemann integral is
not well defined.

Let us now return to the specific example in~\eqref{eq:rational-indicator}.
Since each subinterval $[a_{i-1}, a_i]$ contains both rational and
irrational numbers, we must have that $F_i = 1$ and $f_i = 0$ for every
subinterval.  Thus we have
%
\begin{align*}
  U_n &= \sum_{i=1}^n (a_i - a_{i-1})(1) = a_n - a_0 = 1\\
  L_n &= \sum_{i=1}^n (a_i - a_{i-1})(0) = 0
\end{align*}
%
Since this is true for all partitions of $[0, 1]$, refining the
partition does not change these values.  Therefore we have that
$U = 1$ and $L = 0$, so that $U \neq L$, and, thus, this integral is not
well defined in the Riemannian sense.

On the other hand, suppose that we could find the length or
\textbf{measure} $M$ of the set $S$ of rational numbers between 0 and 1.
Then, for $x \in [0, 1]$, the above fuction $f(x)$ has the value
$f(x) = 0$ on a set of measure $M$, and it has the value $f(x) = 1$ on
the remaining set of measure $1 - M$.  As a result, we could write
%
\begin{equation*}
  I = (M)(0) + (1-M)(1) = 1 - M
\end{equation*}
%

It turns out (as will be discussed in the next section) that the
Riemann measure of $S$ is not well defined, and, thus, neither is the
Riemann integral.  However, we will show in section 1C that the
\textbf{Lebesgue} measure of $S$ is well defined and is given by
$M = 0$.\footnote{
  In some sense, there are very few rational numbers, compared to the
  set of irrational numbers.}
Therefore, the \textbf{Lebesgue} integral exists and has the value
%
\begin{equation*}
  I = 1 - M = 1
\end{equation*}
%

Lebesgue measure and Lebesgue integration are generalizatinos of their
more familiar Riemann counterparts.  If a set has Riemann measure $M$,
then it also has Lebesgue measure $M$; if a Riemann integral has the
value $I$, then Lebesgue integration yields the same value $I$.
However, many sets that have no well defined Riemann measure do have a
well defined Lebesgue measure; and many integrals that are not
Riemann-integrable are Lebesgue-integrable.  Thus, Lebesgue measure
and integration simply extend the more familiar Riemann concepts.
%
\section{Riemann Measure}
%
Riemann defines the measure of a closed interval $[a, b]$ to be
$M = b - a$.  Letting $b = a$, we see that an individual point has Riemann
measure zero.  The Riemann measure is then extended to more
complicated sets via two rules
\begin{enumerate*}
\item
  if the sets $S_1, S_2, \cdots, S_n$ are disjoint and have
  measures $M_1, M_2, \cdots, M_n$, respectively, then their union
  $S = S_1 \cup S_2 \cup \cdots \cup S_n$ has measure
  $M = M_1 + M_2 + \cdots + M_n$
\item
  if $S$ is a set of measure $M$ and $S'$ is a subset of measure
  $M'$, then the set $S - S'$ (i.e., the set of points in $S$
  which are not in $S'$) has measure $M - M'$.  In particular, these
  rules allow one to show that the non-closed intervals
  $[a, b]$, $(a, b]$, and $(a, b)$ all have measure $M = b - a$ as well.
\end{enumerate*}

The above rules allow us to find the Riemann measure of any finite
collection of intervals (including individual points) with or without
endpoints; the measure is just the sum of the lengths of the
intervals.  Furthermore, these are the only sets with are
Riemann-measurealbe (i.e., have a well defined Riemann measure).

\section{Lebesgue Measure}
%
Lebesgue extended the Riemann measure by modifying the first of the
two rules to allow not just finite unions, but also
\textbf{countably infinite} unions\footnote{
  A set $S$ is said to be countably infinite, or just
  countable, if there exists a one-to-one correspondence between the
  set $S$ and the positive integers $\mathbb{Z_+} = \{1, 2,
  \cdots\}$}.
Thus, if the sets $S_1, S_2, \dots$, are disjoint and have measures
$M_1, M_2, \dots$, respectively, then the union
$S = \cup_{i = 1}^\infty S_i$ has measure $M = \sum_{i = 1}^\infty
M_i$.
This simple change greatly increases the collection of measurable
sets, without altering the measure of the Riemann-measureable sets.

As an example, consider the following set, a modified Cantor set.
Beginning with the interval $[0, 1]$, remove the middle third to leave
the two intervals $[0, \sfrac{1}{3}]$ and $[\sfrac{2}{3}, 1]$.  Next,
remove the middle ninth of each of these two intervals to leave the
four intervals
$[0, \sfrac{4}{27}], [\sfrac{5}{27}, \sfrac{1}{3}]$,
$[\sfrac{2}{3}, \sfrac{22}{27}]$,
and $[\sfrac{23}{27}, 1]$.  Continuing this process infinitely many
times leads to some limiting set $S$.  Note that at the $n$th stage of
construction, we remove the middle fraction $f_n = \sfrac{1}{3^n}$ of
each remaining interval.  To better visualize the set $S$, notice that
the $n$th stage of construction resulted in $N_n = 2^n$ intervals of
equal length $L_n$.  Since this must be a subset of the original
interval $[0, 1]$, its total length cannoth exceed 1, which yields
$N_nL_n \leq$ or $L_n \leq \sfrac{1}{N_n} = \sfrac{1}{2^n}$.  Letting
$n \rightarrow \infty$, we have that $L_n \rightarrow 0$, which shows
that the limiting set $S$ contains no intervals of finite length; it
is a completely disconnected set called a dust\footnote{Between any
  two points in $S$ there is a point which is not in S.}.

To compute the Lebesgue measure $M$ of the set $S$, we consider its
complement $S' = [0, 1] - S$, which is just the collection of
intervals removed at the various stages of construction.  Since this
is a countable collection of intervals, its Lebesgue measure $M'$ is
just the sum of the lengths of these intervals.  At the first stage of
construction, we removed $N_0 = 1$ interval of length
$l_0 = \sfrac{1}{3}$.  At the next stage, we removed $N_1 = 2$
intervals of length $l_1 = \sfrac{1}{27}$.  At the next stage, we
removed $N_2 = 4$ intervals of length
$l_2 = \sfrac{1}{27}\left(\sfrac{4}{27}\right)$.  Thus, the measure of
the set consisting of all the removed intervals is
%
\begin{equation*}
  M' = \sum_{n=0}^\infty N_n l_n =
  \left(1\right)\left(\frac{1}{3}\right) +
  \left(2\right)\left(\frac{1}{27}\right) +
  \left(4\right)\left(\frac{4}{729}\right) + \cdots \approx 0.44
\end{equation*}
%
Hence, the dust $S = [0, 1] - S'$ has measure
%
\begin{equation*}
  M = 1 - M' \approx 0.56
\end{equation*}
%
which is finite!  Sets such as these appear in the studies of
turbulence and are referred to as fat fractals.

Note that any countable set of points
$S = \{P_1, P_2, \dots\}$ must have Lebesgue measure zero.
This is because each individual point $P_i$ has Lebesgue measure
$M_i = 0$, and the countable union $S$ of these points has Lebesgue
measure $M = \sum_{i=1}^\infty M_i = 0$.  We will see in section 1F
that the rational numbers are a countable set, so that they
must have Lebesgue measure zero (as used in section 1A).

Finally, we say that something happens \textbf{almost everywhere}
if it happens everwhere except on a set of Lebesgue measure zero.  As
an example, the indicator function defined in section 1A is equal to 1
almost everywhere.
%
\section{Lebesgue Integration}
%
In general, a function can only be integrated over measurable sets.
For Riemann integration, the integral $\int_{S} dx\, f(x)$ of a function
$f(x)$ over a set $S$ exists if and only if $S$ may be written as a
finite collection of disjoint intervals and points such that $f(x)$ is
continuous on each of these intervals.  In particular, a
Riemann-integrable function can have at most a finite number of
discontinuities (unlike the indicator function discussed in section 1A).

For Lebesgue integration $\int_S dx \, f(x)$, it is sufficient to
write $S$ as a countable collection of disjoint intervals and points
such that $f(x)$ is continuous on each of these intervals.  Thus a
Lebesgue-integrable function can have countably many discontinuities.
%
\section{Covering}
%
\section{Countability}
%
\section{Measure Theory}
%
\section{Examples of Measures}
%
\section{Important Theorems}
%
\section{Hausdorff Measure}


