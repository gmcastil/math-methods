

\chapter{Distribution Theory}
\label{ch:distribution-theory}

\section{The Dirac $\delta$-Function}
\label{sec:delta-function}
The $\delta$-function can be defined in several different ways.  The
simplest is to say that
\begin{equation*}
  \delta(x) = \left\{
    \begin{array}{lr}
      0 & : x \neq 0\\
      \infty & : x = 0
    \end{array}
  \right.
\end{equation*}
with $\delta(0)$ being infinite in such a way that
\begin{equation*}
  \intab{a}{b}{x}\delta(x) = 1
\end{equation*}
for $a < 0 < b$ (since $\delta(x) = 0$ for $x \neq 0$ this integral is
independent of $a$ and $b$, so long as $a < 0 < b$).

A second possibility is to define $\delta(x)$ to be the limit as $n
\rightarrow \infty$ of a delta sequence $\delta_n(x)$ with the understanding
that, if there is an integral over $x$, the integration is to be performed
prior to taking the limit as $n \rightarrow \infty$.  There are many possible
delta sequences, one of which is
\begin{equation*}
  \delta_n(x) = \pi^{-\frac{1}{2}} n e^{-n^2x^2}
\end{equation*}
Defining $\delta(x) = \lim_{n \to \infty} \delta_n(x)$ yields
\begin{equation*}
  \delta(x) = \left\{
    \begin{array}{lr}
      0 & : x \neq 0\\
      \infty & : x = 0
    \end{array}
  \right.
\end{equation*}
as before.  In addition, we have (since the integral is to be done before
taking the limit)
\begin{align*}
  \intreal{x} \delta(x) &= \inflim{n} \intreal{x} \delta_n(x) \\
                        &= \inflim{n}
                          \invnroot{2}{\pi} n \intreal{x} e^{-n^2x^2} \\
                        &= \inflim{n}
                          \invnroot{2}{\pi} n
                          \left(\frac{\pi^{\frac{1}{2}}}{n}\right) \\
                        &= \inflim{n} \left(1\right) \\
                        &= 1
\end{align*}
and since $\delta(x) = 0$ for $x \neq 0$, this yields
\begin{equation*}
  \intab{a}{b}{x} \delta(x) = 1
\end{equation*}
whenever $a < 0 < b$, as before.  In fact, using delta sequences, we can show
that for any function $\phi(x)$, we have that
\begin{equation*}
  \intab{a}{b}{x} \phi(x) \delta(x) = \phi(0)
\end{equation*}
whenever $a < 0 < b$.

Finally, $\delta(x)$ can be defined by requiring that for any function
$\phi(x)$, we have that
\begin{equation*}
  \intreal{x} \phi(x)\delta(x) = \phi(0)
\end{equation*}
Furthermore, it can be shown that this yields
$\int_z^bdx\, \phi(x)\delta(x) = \phi(0)$ whenever $a < 0 < b$.

The first of these three definitions is rather intuitive, but it is not very
useful.  To show this, let us attempt to compute the derivative of the product
$f(x)\delta(x)$ for an arbitrary function $f(x)$.  We have
\begin{equation*}
  \ddx{x}\left[f(x)\delta(x)\right] = f'(x)\delta(x) + f(x)\delta'(x)
\end{equation*}
Since $\delta(x) = 0$ for $x \neq 0$, then we must also have $\delta'(x) = 0$
for $x \neq 0$.  Thus, this becomes
\begin{equation*}
  \ddx{x} \left[f(x)\delta(x)\right] = f'(0)\delta(x) + f(0)\delta'(x)
\end{equation*}
However, the same argument yields
\begin{equation*}
  f(x)\delta(x) = f(0)\delta(x)
\end{equation*}
and differentiating this gives
\begin{equation*}
  \ddx{x} \left[f(x)\delta(x)\right] = f(0)\delta'(x)
\end{equation*}
Clearly, something is wrong, but this simplistic definition is unable to
provide the answer.

The second definition in terms of delta sequences is better, but is also unable
to solve the above problem.  In factd, it yields
\begin{align*}
  \delta'(x) &= \inflim{n} \delta_n'(x) \\
             &= \inflim{n}
               \D{}{x} \left(\invnroot{2}{\pi} n e^{-n^2x^2}\right) \\
             &= \inflim{n} (-2\invnroot{2}{\pi} n^3 x e^{-n^2x^2} \\
             &= 0
\end{align*}
for all $x$.  However, we also have
\begin{align*}
  \delta''(x) &= \inflim{n} \delta_n''(x) \\
              &= \inflim{n} \D[2]{}{x}
                \left(\invnroot{2}{\pi} n e^{-n^2x^2}\right) \\
              &= \inflim{n} (-2 \invnroot{2}{\pi} n^3 e^{-n^2x^2}
                + 4 \invnroot{2}{\pi} n^5 x^2 e^{-n^2x^2}) \\
              &= \left\{\begin{array}{cr}
                   0       & : x \neq 0 \\
                   -\infty & : x = 0
                 \end{array}\right.
\end{align*}

But, if $\delta'(x) = 0$, we should have $\delta''(0) = 0$ also, so something
is again wrong.

The problem here is that the delta function is not really a function.  It
cannot be defined by specifying its value at all points.  The third definition
given earlier avoids this problem and is an acceptable definition.  We will
return to this definition in the next section.

\section{Distributions}
\label{sec:distributions}
As mentioned in the previous section, the delta function is not really a
function; it is a distribution.  A distribution is a functional - it is
specified by giving its values for all functions $\phi(x)$, rather than for all
points $x$.  More formally, the delta function is the distribution
\begin{equation*}
  \delta[\phi] = \phi(0)
\end{equation*}

Normal functions are special cases of distributions: as a distribution, a
function $f(x)$ is defined as
\begin{equation*}
  f[\phi] = \intreal{x} \phi(x) f(x)
\end{equation*}
Because of this, for a general distribution $D[\phi]$, one often uses the
notation
\begin{equation*}
  D[\phi] = \intreal{x} \phi(x) D(x)
\end{equation*}
and refers to the `function' $D(x)$.  In this notation, the above definition of
the delta function becomes
\begin{equation*}
  \intreal{x} \phi(x) \delta(x) = \phi(0)
\end{equation*}
which is just the third definition from Section~\ref{sec:delta-function}.  In
the next section, we will show how this definition solves the problem of
differentiating $f(x)\delta(x)$.  The main point of this section is that, for a
general distribution, $D(x)$ is defined only when an integral over $x$ is
present (i.e., only the quantity
\begin{equation*}
  D[\phi] = \intreal{x} \phi(x) D(x)
\end{equation*}
is really defined - $D(x)$ itself is not defined).
\section{Derivatives of Distributions}
A function $f(x)$ corresponds to the distribution
\begin{equation*}
  f[\phi] = \intreal{x} \phi(x) f(x)
\end{equation*}
Similarly, its derivative $f'(x)$ corresponds to the distribution
\begin{equation*}
  f'[\phi] = \intreal{x} \phi(x) f'(x)
\end{equation*}
Integrating by parts in this last equation yields
\begin{align*}
  f'[\phi] &= -\intreal{x} \phi'(x) f(x) \\
           &= -f[\phi']
\end{align*}
Note that we have assumed that $\phi(x) \rightarrow 0$ as $x$ approaches
$\pm \infty$.  This is because the class of functions $\phi(x)$ on which
distributions are defined must satisfy certain restrictions.  In particular,
$\phi(x)$ and all of its derivatives must exist, be continuous, and vanish as
$x$ approaches $\pm \infty$.

The above result motivates the definition of the derivative of a general
distribution as
\begin{equation*}
  D'[\phi] = -D[\phi']
\end{equation*}
or, in terms of the `function' $D(x)$,
\begin{equation*}
  \intreal{x} \phi(x) D'(x) = -\intreal{x} \phi'(x) D(x).
\end{equation*}
An important point is that, like the distribution itself, the derivative of a
general distribution is defined only inside of an integral.

We are now prepared to return to the problem of differentiating
$f(x)\delta(x)$.  By definition, or by integration by parts, we have
\begin{align*}
  \intreal{x} \phi(x) \D{}{x} \left[f(x)\delta(x)\right]
  &= -\intreal{x} \phi'(x) \left[f(x)\delta(x)\right] \\
  &= -\intreal{x} \left[f(x) \phi'(x)\right] \delta(x) \\
  &= -f(0) \phi'(0) \\
  &= -f(0) \intreal{x} \phi'(x) \delta(x) \\
  &= f(0) \intreal{x} \phi(x) \delta'(x) \\
  &= \intreal{x} \phi(x) \left[f(0) \delta'(x)\right]
\end{align*}
Since this must be true for any function $\phi(x)$ we must have
\begin{equation*}
  \D{}{x} \left[f(x)\delta(x)\right] = f(0)\delta'(x)
\end{equation*}
Thus the second calculation in Section~\ref{sec:delta-function} gave the
correct answer.  The problem with the first calculation in
Section~\ref{sec:delta-function} is that
$f(x)\delta'(x) \neq f(0)\delta'(x)$.  In fact, we have that
\begin{align*}
  \intreal{x} \phi(x)\left[f(x)\delta'(x)\right]
  &= \intreal{x} \left[f(x)\phi(x)\right] \delta'(x) \\
  &= -\intreal{x} \left[f'(x)\phi(x) + f(x)\phi'(x)\right] \delta(x) \\
  &= -\left[f'(0)\phi(0) + f(0)\phi'(0)\right] \\
  &= -f'(0) \intreal{x} \phi(x) \delta(x)
                 + f(0) \intreal{x} \phi(x) \delta'(x) \\
  &= \intreal{x} \phi(x) \left[f(0) \delta'(x) - f'(0)\delta(x)\right]
\end{align*}
Hence, we must have that
\begin{equation*}
  f(x)\delta'(x) = f(0)\delta'(x) - f'(0)\delta(x).
\end{equation*}
If this result is used in the first calculation of
Section~\ref{sec:delta-function}, the correct value is again obtained.

Note the following subtle point: $f(x)\delta'(x)$ depends only on the
properties of $f(x)$ at $x = 0$, but it depends on more than just the value of
$f(x)$ at $x = 0$.  The same is true for higher-order derivatives.  In fact, we
have that
\begin{equation*}
  f(x)\delta^{(n)}(x) = \sum_{m=0}^n \left(-1\right)^m \frac{n!}{m!(n-m)!}
                                              f^{(m)}(0) \delta^{(n-m)}(x)
\end{equation*}
\section{Derivatives of Delta Functions}

\section{The Distribution $\delta[f(x)]$}

\section{Products of Functions and Delta Functions}

\section{Products of Distributions}

\section{A Final Example}


%%% Local Variables:
%%% TeX-master: "body"
%%% End:
