% This file is included in body.tex

\chapter{Measure Theory}
\label{ch:measure-theory}
%
\section{Integration and Measure}
%
Consider the indicator function
%
\begin{equation*}
  f(x) = \left\{
    \begin{array}{lr}
      0 & : x \in \mathbb{Q}\\
      1 & : x \not\in \mathbb{Q}
    \end{array}
  \right.
\end{equation*}
%
where $\mathbb{Q}$ represents the set of rational numbers (i.e., those numbers that can be written as a fraction $\sfrac{p}{q}$ with $p$ and $q$ both integers and $q \neq 0$, or equivalently, those numbers that can be written as either terminating or repeating decimals.).  Our goal is to evaluate integrals of the form
%
\begin{equation}
  \label{eq:rational-indicator}
  I = \int_0^1 dx\, f(x)  
\end{equation}
%
The standard approach via Riemann integration is to partition the interval $[0,1]$ into a finite number of subintervals
%
\begin{equation*}
  [0 = a_0, a_1], [a_1, a_2], \cdots, [a_{n-1}, a_n = 1]
\end{equation*}
%
Letting $F_i$ and $f_i$ denote the largest and smallest values of the function $f(x)$ respectively on the $i$th subinterval $[a_{i-1}, a_i]$, we then form the upper and lower sums
%
\begin{align*}
  U_n &= \sum_{i=1}^n (a_i - a_{i-1})F_i\\
  L_n &= \sum_{i=1}^n (a_i - a_{i-1})f_i
\end{align*}
%
As this partition is refined (and $n$ increases), $U_n$ decreases to a limiting value $U$ and $L_n$ increases to a limiting value $L$.  If $U = L$, then the integral is well defined (in the Riemann sense) and has the value $I = U = L$.  If $U \neq L$, then the Riemann integral is not well defined.

Let us now return to the specific example in~\eqref{eq:rational-indicator}.  Since each subinterval $[a_{i-1}, a_i]$ contains both rational and irrational numbers, we must have that $F_i = 1$ and $f_i = 0$ for every subinterval.  Thus we have
%
\begin{align*}
  U_n &= \sum_{i=1}^n (a_i - a_{i-1})(1) = a_n - a_0 = 1\\
  L_n &= \sum_{i=1}^n (a_i - a_{i-1})(0) = 0
\end{align*}
%
Since this is true for all partitions of $[0, 1]$, refining the partition does not change these values.  Therefore we have that $U = 1$ and $L = 0$, so that $U \neq L$, and, thus, this integral is not well defined in the sense of Riemann integration.

On the other hand, suppose that we could find the length or \emph{measure} $M$ of the set $S$ of rational numbers between 0 and 1.  Then, for $x \in [0, 1]$, the above fuction $f(x)$ has the value $f(x) = 0$ on a set of measure $M$, and it has the value $f(x) = 1$ on the remaining set of measure $1 - M$.  As a result, we could write
%
\begin{equation*}
  I = (M)(0) + (1-M)(1) = 1 - M
\end{equation*}
%

It turns out (as will be discussed in the next section) that the Riemann measure of $S$ is not well defined, and, thus, neither is the Riemann integral.  However, we will show in section (cite) that the Lebesgue measure of $S$ is well defined and is given by $M = 0$.\footnote{In some sense, there are very few rational numbers, compared to the set of irrational numbers.}  Therefore the Lebesgue integral exists and has the value
%
\begin{equation*}
  I = 1 - M = 1
\end{equation*}
%

Lebesgue measure and Lebesgue integration are generalizatinos of their more familiar Riemann counterparts.  If a set has Riemann measure $M$, then it also has Lebesgue measure $M$; if a Riemann integral has the value $I$, then Lebesgue integration yields the same value $I$.  However, many sets that have no well defined Riemann measure do have a well defined Lebesgue measure; and many integrals that are not Riemann-integrable are Lebesgue-integrable.  Thus, Lebesgue measure and integration simply extend the more familiar Riemann concepts.
%
\section{Riemann Measure}
%
Riemann defines the measure of a closed interval $[a, b]$ to be $M = b - a$.  Letting $b = a$, we see that an individual point has Riemann measure zero.  The Riemann measure is then extended to more complicated sets via two rules
\begin{enumerate*}
\item if the sets $S_1, S_2, \cdots, S_n$ are disjoint and have measures $M_1, M_2, \cdots, M_n$, respectively, then their union $S = S_1 \cup S_2 \cup \cdots \cup S_n$ has measure $M = M_1 + M_2 + \cdots + M_n$
\item if $S$ is a set of measure $M$ and $S'$ is a subset of measure $M\prime$, then the set $S - S'$ (i.e., the set of points in $S$ which are not in $S'$) has measure $M - M'$.  In particular, these rules allow one to show that the non-closed intervals $[a, b]$, $(a, b]$, and $(a, b)$ all have measure $M = b - a$ as well.
\end{enumerate*}
The above rules allow us to find the Riemann measure of any finite collection of intervals (including individual points) with or without endpoints; the measure is just the sum of the lengths of the intervals.  Furthermore, these are the only sets with are Riemann-measurealbe (i.e., have a well defined Riemann measure).